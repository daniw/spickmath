% coding:utf-8

%----------------------------------------
%FOSAMATH, a LaTeX-Code for a mathematical summary for basic analysis
%Copyright (C) 2013, Daniel Winz, Ervin Mazlagic, Adrian Imboden, Philipp Langer

%This program is free software; you can redistribute it and/or
%modify it under the terms of the GNU General Public License
%as published by the Free Software Foundation; either version 2
%of the License, or (at your option) any later version.

%This program is distributed in the hope that it will be useful,
%but WITHOUT ANY WARRANTY; without even the implied warranty of
%MERCHANTABILITY or FITNESS FOR A PARTICULAR PURPOSE.  See the
%GNU General Public License for more details.
%----------------------------------------

% coding:utf-8
\section{Aufgabe zu Implizitem Ableiten}
Implizite Gleichungen bzw. Funktionen sind Funktionen welche die Form $x^a + d x y + y^b + c= 0$ haben.
Gegenüber der expliziten Form $f(x)=y=\ldots$ ist $y$ selbst als Funktion von $x$ definiert.\\\\
Bsp.: Aus einer implizit formulierten Gleichung eine Tangente an einem bestimmten Punkt legen.
\[ x^a + d x y + y^b + c = 0, \quad T_{(x_0)} = f'(x_0)(x-x_0)+f(x_0), \quad P(x_p|y_p) \]
Zuerst muss der Ausdruck $y$ durch $y(x)$ ersetzt werden.
\[ x^a + d x y + y(x)^b = 0\]
Danach kann abgeleitet und nach $y'(x)$ aufgelöst werden.
\[ \begin{array}{lll@{\quad}}
ax^{a-1} + d\Big(y(x) + xy'(x)\Big) + y'(x)by(x)^{b-1} =0 & |& \text{ausklammern} \\\\
ax^{a-1} + dy(x) + axy'(x) + y'(x)by(x)^{b-1} =0          & |& -ax^{a-1} - dy(x) \\\\
dxy'(x) + y'(x) by(x)^{b-1} = -ax^{a-1} - dy(x)           & |& y'(x) \text{ ausklammern } \\\\
y'(x) \cdot (dx + by^{b-1}) = -ax^{a-1} - dy(x)           & |& \div (dx+by^{b-1}) \\\\
y'(x) = \dfrac{-ax^{a-1} - dy(x)}{dx + by^{b-1}}          &  & \\\\
\end{array} \]
Nun kann man die Tangentengleichung bilden:
\[ T(x) = y'(x_p) (x-x_p) + y(x_p) \]
