% coding:utf-8

%----------------------------------------
%FOSAMATH, a LaTeX-Code for a mathematical summary for basic analysis
%Copyright (C) 2013, Daniel Winz, Ervin Mazlagic, Adrian Imboden, Philipp Langer

%This program is free software; you can redistribute it and/or
%modify it under the terms of the GNU General Public License
%as published by the Free Software Foundation; either version 2
%of the License, or (at your option) any later version.

%This program is distributed in the hope that it will be useful,
%but WITHOUT ANY WARRANTY; without even the implied warranty of
%MERCHANTABILITY or FITNESS FOR A PARTICULAR PURPOSE.  See the
%GNU General Public License for more details.
%----------------------------------------

% coding:utf-8
\section{Aufgabe zu Taylorreihen}
Gegeben ist die Funktion 
\[ f(x) = \frac{x-1}{(3+x)^2} \]
a)\\
Berechnen Sie das Taylorpolynom 3. Grades von f(x) um den Punkt x=1 mit dem Taschenrechner
\[ f(x) = \frac{x - 1}{(3 + x)^2}\quad , x_0 = 1 \]
\begin{enumerate}
  \item Approximationspunkt bestimmen ($x_0 = 1$)
  \item Grad (3. Grad)
  \item Glieder berechnen
\end{enumerate}
\[ f(x) \]
\[ f'(x) = \frac{(x - 1)' \cdot (3 + x)^2 - (x - 1) \cdot (3 + x)^2}{((3 + x)^2)^3} = \frac{1 \cdot (3 + x)^2 - (x - 1) \cdot 2 \cdot (3 + x) \cdot 1}{(3 + x)^4} \]
\[ = \frac{(3 + x)^2 - (x - 1) \cdot 2 \cdot (3 + x)}{(3 + x)^4} = \frac{(3 + x) - 2 \cdot (x - 1)}{(3 + x)^3} = \frac{3 + x - 2x + 2}{(3 + x)^3} = \frac{5 - x}{(3 + x)^3} \]
\[ f''(x) = \frac{(5 - x)' \cdot (3 + x)^3 - (5 - x)\cdot((3 + x)^3)'}{((3 + x)^3)^2} = \frac{(-1) \cdot (3 + x)^3 - (5 - x)\cdot3(3 + x)^2}{(3 + x)^6} \]
\[ = \frac{(-1) \cdot (3 + x) - (5 - x) \cdot 3}{(3 + x)^4} = \frac{-3 - x - 15 + 3x}{(3 + x)^4} = \frac{2x - 18}{(3 + x)^4} \]
\[ f'''(x) = \frac{}{} \]
\[  \]
\[  \]
\[  \]
\[  \]
\[  \]
\[  \]
\[  \]
\[  \]
b)\\
Bestimmen Sie mit Hilfe von a) näherungsweise das Integral $\int_0^1 f(x) dx$. Runden Sie das Resultat bis auf 3 Stellen nach dem Komma und vergleichen Sie es mit dem direkten Resultat des Tschenrechners. 
\[  \]
\[  \]
\[  \]
c)\\
Verwenden Sie eine Binomische Reihe um die Taylorreihe von $f(x)$ um $x=1$ zu berechnen. Hinweis: entwickeln Sie $f(1+h)$ in eine MacLaurin Reihe
\[  \]
\[  \]
\[  \]