% coding:utf-8

%----------------------------------------
%FOSAMATH, a LaTeX-Code for a mathematical summary for basic analysis
%Copyright (C) 2013, Daniel Winz, Ervin Mazlagic, Adrian Imboden, Philipp Langer

%This program is free software; you can redistribute it and/or
%modify it under the terms of the GNU General Public License
%as published by the Free Software Foundation; either version 2
%of the License, or (at your option) any later version.

%This program is distributed in the hope that it will be useful,
%but WITHOUT ANY WARRANTY; without even the implied warranty of
%MERCHANTABILITY or FITNESS FOR A PARTICULAR PURPOSE.  See the
%GNU General Public License for more details.
%----------------------------------------

% coding:utf-8
\section{Aufgabe zu Taylorreihen}
Gegeben ist die Funktion 
\[ f(x) = \frac{x-1}{(3+x)^2} \]
a)\\
Berechnen Sie das Taylorpolynom 3. Grades von f(x) um den Punkt x=1 mit dem Taschenrechner
\[ f(x) = \frac{x - 1}{(3 + x)^2}\quad , x_0 = 1 \]
\begin{enumerate}
  \item Approximationspunkt bestimmen ($x_0 = 1$)
  \item Grad (3. Grad)
  \item Glieder berechnen
\end{enumerate}
\[ f(x) \]
\[ f'(x) = \frac{(x - 1)' \cdot (3 + x)^2 - (x - 1) \cdot (3 + x)^2}{((3 + x)^2)^3} = \frac{1 \cdot (3 + x)^2 - (x - 1) \cdot 2 \cdot (3 + x) \cdot 1}{(3 + x)^4} \]
\[ = \frac{(3 + x)^2 - (x - 1) \cdot 2 \cdot (3 + x)}{(3 + x)^4} = \frac{(3 + x) - 2 \cdot (x - 1)}{(3 + x)^3} = \frac{3 + x - 2x + 2}{(3 + x)^3} = \frac{5 - x}{(3 + x)^3} \]
\[ f''(x) = \frac{(5 - x)' \cdot (3 + x)^3 - (5 - x)\cdot((3 + x)^3)'}{((3 + x)^3)^2} = \frac{(-1) \cdot (3 + x)^3 - (5 - x)\cdot3(3 + x)^2}{(3 + x)^6} \]
\[ = \frac{(-1) \cdot (3 + x) - (5 - x) \cdot 3}{(3 + x)^4} = \frac{-3 - x - 15 + 3x}{(3 + x)^4} = \frac{2x - 18}{(3 + x)^4} \]
\[ f'''(x) = \frac{2 \cdot (3 + x)^4 - (2x - 18) \cdot 4 \cdot (3 + x)^3}{((3 + x)^4)^2} = \frac{(3 + x)^3 \cdot (2 \cdot (3 + x) - 4 \cdot (2x - 18))}{(3 + x)^8} \]
\[ = \frac{2 \cdot (3 + x) - 4 \cdot (2x - 18)}{(3 + x)^5} = \frac{6 + 2x - 8x + 18 \cdot 4}{(3 + x)^5} = \frac{-6x + 78}{(3 + x)^5} \]
Taylorreihe zusammensetzen
\[ T_{(3, x_0)}(x) = \frac{f(x_0)}{0!}(x - x_0)^0 + \frac{f'(x_0)}{1!}(x - x_0)^1 + \frac{f''(x_0)}{2!}(x - x_0)^2 + \frac{f'''(x_0)}{3!}(x - x_0)^3 \]
\[ T_{(3, x_0)}(x) = \frac{\frac{1-1}{(3+1)^2}}{0!}(x - 1)^0 + \frac{\frac{5 - 1}{(3 + 1)^3}}{1!}(x - 1)^1 + \frac{\frac{2\cdot1 - 18}{(3 + 1)^4}}{2!}(x - 1)^2 + \frac{\frac{-6\cdot1 + 78}{(3 + 1)^5}}{3!}(x - 1)^3 \]
\[ T_{(3, x_0)}(x) = 0 + \frac{\frac{4}{(4)^3}}{1!}(x - 1)^1 + \frac{\frac{-16}{(4)^4}}{2!}(x - 1)^2 + \frac{\frac{72}{(4)^5}}{3!}(x - 1)^3 \]
\[ T_{(3, x_0)}(x) = \frac{\frac{1}{(4)^2}}{1!}(x - 1)^1 + \frac{-\frac{1}{(4)^2}}{2!}(x - 1)^2 + \frac{\frac{18}{(4)^4}}{3!}(x - 1)^3 \]
\[ T_{(3, x_0)}(x) = \frac{\frac{1}{16}}{1}(x - 1)^1 + \frac{-\frac{1}{16}}{2}(x - 1)^2 + \frac{\frac{18}{256}}{6}(x - 1)^3 \]
\[ \underline{\underline{T_{(3, x_0)}(x) = \frac{1}{16}(x - 1)^1 - \frac{1}{32}(x - 1)^2 + \frac{3}{256}(x - 1)^3}} \]
% b)\\
% Bestimmen Sie mit Hilfe von a) näherungsweise das Integral $\int_0^1 f(x) dx$. Runden Sie das Resultat bis auf 3 Stellen nach dem Komma und vergleichen Sie es mit dem direkten Resultat des Tschenrechners. 
% c)\\
% Verwenden Sie eine Binomische Reihe um die Taylorreihe von $f(x)$ um $x=1$ zu berechnen. Hinweis: entwickeln Sie $f(1+h)$ in eine MacLaurin Reihe
